%
\documentclass[12pt,a4paper]{article}

\usepackage[utf8]{inputenc}
\usepackage[usenames,dvipsnames]{xcolor}
\usepackage{graphicx}
\usepackage[margin=1in]{geometry}
\usepackage[]{natbib}
\usepackage[]{sidecap}
\usepackage{setspace}
\usepackage{defs}
\usepackage{lastpage}
\usepackage{fancyhdr}
\usepackage{pdfpages}
\usepackage{enumitem}

\usepackage{libertine}
\usepackage[T1]{fontenc}

\usepackage[pdftex]{hyperref}
\hypersetup {
    bookmarks=true,                    % show bookmarks bar in pdf reader
    pdftitle={Research Proposal - 2016 VR Etableringsbidrag},                   
    pdfauthor={Gregory A. Feiden},     % set pdf author
    pdfsubject={}, % pdf subject
    colorlinks=true,                   % false = box link, true = colored links
    linkcolor=black,                   % color of internal links
    citecolor=black,                    % color of citations
    urlcolor=NavyBlue                      % external url color
}
\urlstyle{same}

\citestyle{aa}
\bibliographystyle{mn}

\pagestyle{fancy}
\fancyhead[L]{Gregory Feiden -- Personal Number -- Research Plan}
\fancyhead[R]{\thepage\ of\ \pageref{LastPage}}
\fancyfoot[C]{}
\renewcommand{\headrulewidth}{0.0pt}

\fancypagestyle{plain}{
    \fancyhf{}
    \fancyfoot[C]{\thepage\ of\ \pageref{LastPage}}
    \renewcommand{\headrulewidth}{0.0pt}
    \renewcommand{\footrulewidth}{0.0pt}
}

\setlength{\parindent}{0pt}
\setlength{\parskip}{0.5\baselineskip}
\newenvironment{myindentpar}[1]%
 {\begin{list}{}%
         {\setlength{\leftmargin}{#1}}%
         \item[]%
 }
 {\end{list}}

\begin{document}

%\thispagestyle{plain}
\begin{center}
	{\bf {\LARGE Ages of Young Stars and the Evolution of 
	
	Dynamo-Generated Magnetic Fields}} 
	
%	\vspace{0.5\baselineskip}
%	{\it Vetenskapsr\aa det Starting Grant Research Plan, 01 Apr 2016}
%	
%	Gregory A. Feiden
\end{center}
\vspace{\baselineskip}

Stars with masses below 5 solar masses begin their lives

{\bf \large Purpose and Aims}

{\bf \large Survey of the Field}
\begin{itemize}
	\item Age-dating problems for young stars. (i.e., model inconsistencies)
	\item Identification / inclusion of magnetic fields in models. (D'Antona, MacDonald, Feiden)
	\item Magnetic field observations. (e.g., MDI, ZDI)
	\item Open questions.
\end{itemize}

{\bf \large Project Description}

\textbf{\emph{P1. Dynamo-Generated Magnetic Fields throughout Early (sub)Stellar Evolution}}
Trace the temporal evolution of dynamo-generated magnetic fields. Explore how these properties depend on rotation, convection zone size, and stellar fundamental properties (surface temperature). Investigate the (1) topology of large-scale magnetic fields as a function of stellar mass, chemical composition, and age; (2) role of interior magnetic fields in governing stellar structure; and (3) the evolution of stellar activity cycles through polarity reversals of the large-scale field.
% Can we say anything about differential rotation, or is that assumed?

\emph{P1.1 Magnetic Fields in Surface Convection Zones of Low- and Intermediate-Mass Stars} \\
Straight 2D/3D models allow one to explore field properties as a function of stellar properties, but one loses temporal information. Strong magnetic fields are able to inhibit convective motions and delay contraction of young, pre-main-sequence stars. This means stellar rotation periods and convection zone sizes (core and surface) as a function of age differ between for stars with strong magnetic fields and those without. To model the feedback of the magnetic field on stellar properties over stellar evolutionary timescales (hundreds of thousands of years), simpler 1D stellar structure models are necessary. 

\emph{P1.2 Core Convection in the Presence of Magnetic Fields} \\

\emph{P1.3 Magnetic Fields in Young Brown Dwarfs and Giant Planets} \\
Investigating the generation and impact of magnetic fields in cool, high density plasmas. Finite electrical conductivity becomes a significant factor. Seek to answer: Is the contraction time of young brown dwarfs and giant planets affected by magnetic fields?

\textbf{\emph{Magnetic Inhibition of Convection, Starspots, and Atmospheric Properties}} \\
Improved treatment of surface boundary conditions through incorporation of detailed frequency-dependent radiative transfer. Seek to answer: How are surface boundary conditions affected by magnetic inhibition of convection, and how do they affect interior structure? How do starspots affect observational properties of young stars? 

\textbf{\emph{Homogeneous Ages for Young Stellar Associations}}

{\bf \large Significance} \\
Young stellar ages $\rightarrow$ protoplanetary disk evolution timescale thus giant planet formation timescale; star formation history, including mass distribution (IMF); mass distribution of directly imaged planets / brown dwarfs.

{\bf \large Preliminary Results} \\
Feiden (2016, submitted).

Pilot study showing results for the Sun (and a low-mass star?).

{\bf \large Independent Line of Research}

{\bf \large International and National Collaborations}

Observational Campaigns:
\begin{itemize}
	\item Kraus, Rizzuto, Mann (UT Austin) --- B- through M-type EBs in young associations with K2 \& magnetic fields of young stars with IGRINS.
	\item Kochukhov (Uppsala), Hussain (ESO) --- magnetic field topologies (MDI/ZDI) of young stars.
\end{itemize}

{\bf \large Other Grants}

\end{document}
