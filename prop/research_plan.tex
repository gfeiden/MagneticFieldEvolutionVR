%
\documentclass[12pt,a4paper]{article}

\usepackage[utf8]{inputenc}
\usepackage[usenames,dvipsnames]{xcolor}
\usepackage{graphicx}
\usepackage[margin=1in]{geometry}
\usepackage[]{natbib}
\usepackage[]{sidecap}
\usepackage{setspace}
\usepackage{defs}
\usepackage{lastpage}
\usepackage{fancyhdr}
\usepackage{pdfpages}
\usepackage{enumitem}

\usepackage{libertine}
\usepackage[T1]{fontenc}

\usepackage[pdftex]{hyperref}
\hypersetup {
    bookmarks=true,                    % show bookmarks bar in pdf reader
    pdftitle={Research Proposal - 2016 VR Etableringsbidrag},                   
    pdfauthor={Gregory A. Feiden},     % set pdf author
    pdfsubject={}, % pdf subject
    colorlinks=true,                   % false = box link, true = colored links
    linkcolor=black,                   % color of internal links
    citecolor=black,                    % color of citations
    urlcolor=NavyBlue                      % external url color
}
\urlstyle{same}

\citestyle{aa}
\bibliographystyle{mn}

\pagestyle{fancy}
\fancyhead[L]{Gregory Feiden -- Personal Number -- Research Plan}
\fancyhead[R]{\thepage\ of\ \pageref{LastPage}}
\fancyfoot[C]{}
\renewcommand{\headrulewidth}{0.0pt}

\fancypagestyle{plain}{
    \fancyhf{}
    \fancyfoot[C]{\thepage\ of\ \pageref{LastPage}}
    \renewcommand{\headrulewidth}{0.0pt}
    \renewcommand{\footrulewidth}{0.0pt}
}

\setlength{\parindent}{0pt}
\setlength{\parskip}{0.5\baselineskip}
\newenvironment{myindentpar}[1]%
 {\begin{list}{}%
         {\setlength{\leftmargin}{#1}}%
         \item[]%
 }
 {\end{list}}

\begin{document}

%\thispagestyle{plain}
\begin{center}
	{\bf {\LARGE Ages of Young Stars and the Evolution of 
	
	Dynamo-Generated Magnetic Fields}} 
	
%	\vspace{0.5\baselineskip}
%	{\it Vetenskapsr\aa det Starting Grant Research Plan, 01 Apr 2016}
%	
%	Gregory A. Feiden
\end{center}
\vspace{\baselineskip}

Stars with masses below 5 solar masses begin their lives

{\bf \large Purpose and Aims}

{\bf \large Survey of the Field}
\begin{itemize}
	\item Age-dating problems for young stars. (i.e., model inconsistencies)
	\item Identification / inclusion of magnetic fields in models. (D'Antona, MacDonald, Feiden)
	\item Magnetic field observations. (e.g., MDI, ZDI)
	\item Open questions.
\end{itemize}

{\bf \large Project Description}

\textbf{\emph{Dynamo-Generated Magnetic Fields throughout Early Stellar Evolution}}



\textbf{\emph{Magnetic Inhibition of Convection, Starspots, and Atmospheric Properties}} \\
Improved treatment of surface boundary conditions through incorporation of detailed frequency-dependent radiative transfer. Seek to answer: How are surface boundary conditions affected by magnetic inhibition of convection, and how do they affect interior structure? How do starspots affect observational properties of young stars? 

\textbf{\emph{Core Convection in the Presence of Magnetic Fields}}



\textbf{\emph{Magnetic Fields in Young Brown Dwarfs and Giant Planets}} \\
Investigating the generation and impact of magnetic fields in cool, high density plasmas. Finite electrical conductivity becomes a significant factor. Seek to answer: Is the contraction time of young brown dwarfs and giant planets affected by magnetic fields?


{\bf \large Significance} \\
Young stellar ages $\rightarrow$ protoplanetary disk evolution timescale thus giant planet formation timescale; star formation history, including mass distribution (IMF); mass distribution of directly imaged planets / brown dwarfs.

{\bf \large Preliminary Results} \\
Feiden (2016, submitted).

Pilot study showing results for the Sun (and a low-mass star?).

{\bf \large Independent Line of Research}

{\bf \large International and National Collaborations}

Observational Campaigns:
\begin{itemize}
	\item Kraus, Rizzuto, Mann (UT Austin) --- B- through M-type EBs in young associations with K2 \& magnetic fields of young stars with IGRINS.
	\item Kochukhov (Uppsala), Hussain (ESO) --- magnetic field topologies (MDI/ZDI) of young stars.
\end{itemize}

{\bf \large Other Grants}

\end{document}
